% https://tabs.ultimate-guitar.com/tab/misc_traditional/der_piet_chords_2080355

\beginsong{Piet am Galgen}[by={Erik Martin, Traditionell}, index={Was kann ich denn dafür}, index={Der Piet am Galgen hängt}]
\transpose{2}
\capo{2}
\musicnote{original -2 (Am C G / Am D F G)}
% key original Am

\beginverse
Was \[Am]kann ich denn dafür? \brk So \[C]kurz vor meiner Tür,
da \[G]fingen sie mich ein \brk, und bald \[Am]endet meine Pein.
Ich \[Am]hatte niemals Glück. \brk Mein \[C]trostloses Geschick
nahm \[G]keinen von eich ein, \brk ja heut' \[Am]soll gestorben sein.
\endverse

\beginchorus
\lrep \[Am]Wenn der Nebel auf das \brk \[D]Moor sich senkt,
der \[F]Piet am \[G]Galgen \[Am]hängt. \rrep \rep{2}
\endchorus

\beginverse
Sie \[Am]nahmen mir die Schuh \brk und \[C]auch den Rock dazu,
sie \[G]banden mir die Händ' \brk und mein \[Am]Haus, es hat gebrennt.
Ich \[Am]sah den Galgen steh'n, \brk sie \[C]zwangen mich zu geh'n.
Sie \[G]wollten meinen Tod, \brk keiner \[Am]half mir in der Not.
\endverse

\beginchorus \textnote{Refrain} \endchorus

\beginverse
Was \[Am]kratzt da im Genick? \brk Ich \[C]spür' den rauhen Strick.
Ein \[G]Mönch, der betet dort \brk und spricht \[Am]für mich fromme Wort'.
Die \[Am]Wort', die ich nicht kenn', \brk wer \[C]lehrte sie mich denn?
Fünf \[G]Raben fliegen her, \brk doch ich \[Am]sehe sie nicht mehr.
\endverse

\beginchorus \textnote{Refrain} \endchorus

\beginverse
Nun \[Am]bin ich also tot, \brk zu \[C]Ende ist die Not,
am \[G]Himmelstor steh ich, \brk doch der \[Am]Petrus öffnet nicht.
Ach \[Am]Petrus, lass mich ein, \brk ich \[C]will auch artig sein,
Hier \[G]draußen ist es heiß, \brk wofür \[Am]zahl ich denn den Preis?
\endverse

\beginchorus \textnote{Refrain} \endchorus
\endsong
