% Tonspur

\beginsong{Ein stolzes Schiff}[by={Traditionell}]
% key original A

\beginverse
Ein stolzes \[A]Schiff streicht \[D]einsam durch die \[A]Wellen
und führt uns \[Bm]uns're \[E7]deutschen Brüder \[A]fort.
Die Fahne weht, die \[D]weißen Segel \[A]schwellen,
Ameri\[Bm]ka ist \[E7]ihr Bestimmungs\[A]ort.
Seht, auf \[F#m]dem Verdeck sie \[D]stehen,
Sich noch \[F#m]einmal umzu\[D]sehen,
Ins Vater\[A]land, ins heimatliche \[E]Grün.
\[F#m]Seht, wie sie übers \[D]große \[E]Weltmeer \[A]zieh'n.
\endverse

\beginverse
Sie zieh'n da\[A]hin auf \[D]blauen Meeres\[A]wogen.
Warum ver\[Bm]lassen \[E7]sie ihr Heimat\[A]land?
Man hat sie um ihr \[D]Leben schwer be\[A]trogen;
Die Armut \[Bm]trieb sie \[E7]aus dem Vater\[A]land.
Schauet \[F#m]auf, ihr Unter\[D]drücker,
Schauet \[F#m]auf, ihr Volksbe\[D]trüger!
Seht, eure \[A]besten Arbeitskräfte \[E]flieh'n.
\[F#m]Seht, wie sie übers \[D]große \[E]Weltmeer \[A]zieh'n.
\endverse

\beginverse
Sie zieh'n da\[A]hin, wer \[D]wagt sie noch zu \[A]fragen?
Warum ver\[Bm]lassen \[E7]sie ihr Heimat\[A]land?
O, armes Deutschland, wie \[D]kannst du es er\[A]tragen,
Daß deine \[Bm]Brüder \[E7]werden so ver\[A]bannt:
Was sie \[F#m]hofften, hier zu \[D]gründen,
Suchen \[F#m]sie dort drüben zu \[D]finden.
Drum ziehen \[A]sie von deutschem Boden \[E]ab
\[F#m]Und finden in A\[D]meri\[E]ka ihr \[A]Grab.
\endverse

\textnote{Strophe 1}
\endsong
